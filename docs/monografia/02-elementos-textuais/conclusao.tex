% -----------------------------------------------------------------------------
% Conclusão
% -----------------------------------------------------------------------------

\chapter{Conclusão}
\label{chap:conclusao}

O desenvolvimento do presente trabalho possibilitou a avaliação da aplicação de orquestração de cargas de trabalho de análise de dados de prescrições de azitromicina, advindo de uma base consideravelmente grande. Através de profunda revisão bibliográfica foram identificados outras plataformas de orquestração de cargas de trabalhos e foi possível fazer uma avaliação critica baseada nos requisitos disponibilizados nas documentações oficiais, bem como a avaliação crítica do proposito ao qual o presente trabalho se propunha. Ainda na literatura, foi possível encontrar dados e informações a respeito dos impactos socio-econômicos resultantes da restrição orçamentária a pesquisa de uma forma geral. Em específico no caso desse trabalho para análise de dados em saúde, especialmente no auxilio de extração de informações relevantes de grande base de dados. O que torna possível a tomada de decisão em saúde pautada em dados e também auxilia a população a ter melhor noção, baseada em dados, da situação de saúde no qual se encontra e assim poder auditar os órgãos públicos responsáveis pela condução do SUS e políticas de saúde associadas.

Diversas tecnologias foram incorporadas e avaliadas para a orquestração das tarefas de processamento e ingestão de dados a serem utilizadas e \emph{cluster} ubernetes, o que se apresentou viável como alternativa econômica, para analise de grandes volumes de dados utilizando computadores comuns e reaproveitados. A avaliação de performance foi inclusiva, sendo identificados diversos pontos de melhoria e otimizações possiveis o que deixa claro que a integração de profissionias de multiplas especialidades é essencial para utilizar essa alternativa como factível em estudos maiores e mais exigentes. Porém se mostra promissor, especialmente sob o ponto de utilização de recursos subutilizados, com o processo de recrutamento por meio de configuração dinâmica de gerenciadores de configurações. 

Sobre o consumo de azitromicina, há uma influência clara do lançamento do kit COVID, essa relação de prescrições se mostra mais proeminiente e estatisticamente significante em maioria em estados eleitores do governo. O que indica que ações de saúde apresentam um grande viés politico, quando avaliamos a adesão da sociedade. 

É importante que novas estratégias de avaliação de dados em saúde sejam propostas e validadas especialmente na produção e viabilização de uso de ferramentas que contextualizem o cenário de investimento em pesquisa e educação do nosso país a extrapolar a questão de regime politico, afim de que seja viável produzir soluções e pesquisas cada vez mais rapido no campo da saúde, ajudando a comunidade técnica e geral a tomarem melhores descisões embasadas em informações produzidas pela comunidade ciêntifica. 

Restrições orçamentárias não só impactam na velocidade, mas também na viabilização de projetos cientificos que se valem de tecnologias. Na analise e processamento de dados publicos, devido ao volume de dados produzidos no Brasil, isso se torna um impecilio e portanto projetos como esse, porém mais extensos são necessários.

\section{Trabalhos Futuros}
\label{sec:trabalhosFuturos}

O estudo de estratégias de otimização do dimensionamento de recursos e avaliação comparativa de mais tecnologias podem tornar ainda mais plausivel a utilização de computadores desktop subutilizados nas universidades para provisionamento do ambiente de analise de grande massas de dados. Permitindo assim propor estratégias e ferramental indicados como solução para essas analises, bem como uma estratégia viavel de utilização desse pool de computadores ociosos.
% \section{Considerações Finais}
% \label{sec:consideracoesFinais}

% As derradeiras palavras para encerramento do seu trabalho acadêmico.

% -----------------------------------------------------------------------------
% Observação: A norma ABNT estabelece que em qualquer tipo de trabalho
% acadêmico monográfico deve haver um capítulo de conclusão
% -----------------------------------------------------------------------------
