% -----------------------------------------------------------------------------
% Introdução
% -----------------------------------------------------------------------------

\chapter{Introdução}
\label{chap:introducao}

\section{Motivação}
\label{sec:motivacao}

No contexto da análise de dados, diferentes ferramentas estão disponíveis para transformá-los em informação, contudo o uso dessas ferramentas na área da saúde ainda é pouco significativo \cite{galvao_desafios_2019}. Frente a uma tendência crescente de interconexão entre diferentes áreas do conhecimento e do potencial que a análise de dados possibilita para melhoria do sistema de saúde, se faz necessário propor e validar estratégias que permitam o avanço na integração de dados entre diferentes Sistemas de Informação em Saúde (SIS), e que facilitem o processamento e análise do grande volume de dados produzidos e disponibilizados nesses sistemas \cite{galvao_desafios_2019,mehta_concurrence_2018}.

Atualmente, conforme determinação do Decreto nº 8.777, de 11 de maio de 2016, que instituiu a Política de Dados Abertos do Poder Executivo Federal \cite{brasildisponibilidade2016}, diversos dados dos SIS são disponibilizados de forma pública. No entanto, apenas a disponibilização dos dados em si não garante que eles poderão ser analisados visando produzir informação relevante para as políticas públicas na área da saúde. É necessário também a disponibilidade de equipes multidisciplinares que garantam a análise desses dados por meio do uso de ferramentas e recursos adequados. 

\section{Justificativa}
\label{sec:justificativa}


Alterações de governo e vertentes políticas mudam as prioridades de investimento de recursos púlbicos, o que faz com que o orçamento de alguns pilares de investimento, a exemplo das áreas de ciência e tecnologia, sofra importantes flutuações ao longo do tempo (Tabela \ref{tab:orcamento}). Nos anos de 2018 a 2021, o orçamento destinado à pesquisa e desenvolvimento tecnológico no país sofreu reduções acentuadas, o que tornou ainda mais limitado o acesso a recursos que viabilizem a realização de análise dos dados em ferramentas e infraestruturas tradicionais ou ainda a proposta de novas alternativas.

Em uma avaliação simples desses dados, somado a variação do valor do dólar no mesmo periodo, pode-se observar (Figura \ref{fig:dolar}) um aumento de mais de $327\%$. No entanto,  o orçamento não acompanhou essa flutuação. Essa ligação, apesar de não ser direta existe, sendo que em ciência e tecnologia equipamentos, licenças de \emph{software}, entre outros recursos utilizados no processamento de dados, são majoritariamente importadas, e portanto em dólar. Como o valor do orçamento efetivo por ano não acompanha o aumento do dólar, fica claro a perda de poder de compra, dado que a distribuição de verba do orçamento total tenha se mantido. 

Algo que valida a flutuação orçamentária é a queda consecutiva de orçamento desde 2018 (Figura \ref{fig:quedaorca}). Ainda podemos entender que, mesmo depois de 10 anos, temos um orçamento $2,32\%$ menor para ciência e tecnologia. Essa queda na disponibilidade de recursos financeiros impacta diretamente novos projetos, especialmente na compra de equipamentos que viabilizam pesquisas que se valem de bases muito grandes, como as bases de dados do DataSUS e outros sistemas que o compõem. Também pode se exemplificar análises que possuem um maior grau de complexidade como \emph{data linkage} entre bancos de origem distintas como, por exemplo, informações sobre o orçamento da União com informações sobre o pagamento diário e/ou semanal e o banco de despesas do SUS. Sem adequada capacidade computacional à disposição para estudo desses bancos, principalmente em instituições públicas, fica inviável a produção de conhecimento, o monitoramneto do emprego dos recursos públicos pela população e qualquer extração de informação viável. 

Essas limitações indiretas impõem fortes restrições no processo de avaliação e tomada de decisão embasada em dados. Ainda avaliando esse impacto econômico do problema descrito acima, é preciso pensar, como já discutido, a complexidade da tomada de decisões em saúde. Sem informações que embasem essas decisões, fica restrita a capacidade de compreensão do cenário nacional, sobre a situação de saúde pública. Quando leva-se em consideração que o SUS é um sistema subfinanciado, quando comparado a países que também possuem sistemas de saúde pública e gratuitas. O valor do orçamento por dia por pessoa é de apenas $R\$ 3,83$  o que é um preceito quando considera-se  restrição orçamentária do SUS por pessoa. Para entender a complexidade de um sistema tão abrangente é preciso avaliar sua escala. Mais de 152 milhões de pessoas utilizam exclusivamente o SUS como promotor de saúde. 

\begin{table}[htpb]
\centering
  \caption{Pagamento efetivo - Ministério da Ciência e Tecnologia}
  \label{tab:orcamento}
  \begin{tabular}{cc}
    \hline
    \textbf{Ano} & \textbf{Objetivo}    \\
    \hline
    2010         & R\$ 6.288.931.123,00 \\
    2011         & R\$ 5.918.584.706,00 \\
    2012         & R\$ 6.918.288.201,00 \\
    2013         & R\$ 7.787.464.592,00 \\
    2014         & R\$ 8.598.785.224,00 \\
    2015         & R\$ 7.964.319.815,00 \\
    2016         & R\$ 8.404.014.691,00 \\
    2017         & R\$ 9.085.620.227,00 \\
    2018         & R\$ 9.157.748.260,00 \\
    2019         & R\$ 8.812.096.752,00 \\
    2020         & R\$ 7.859.851.948,00 \\
    2021         & R\$ 6.142.873.884,00
  \end{tabular}
  \centering
  \begin{tablenotes}
  \centering
    \small
    \item{Fonte: SIOP consulta realizada em Janeiro de 2022}
  \end{tablenotes}
\end{table}

\begin{figure}[!h]
    \centering
    \includegraphics[width=0.8\textwidth]{04-figuras/dolar.png}
    \caption[Evolução do dólar/real na última década ]{Evolução do dólar/real na última década (Fonte: Feito pelo autor)}
    \label{fig:dolar}
\end{figure}
\begin{figure}[!h]
    \centering
    \includegraphics[width=0.8\textwidth]{04-figuras/compartive.png}
    \caption[Pagamento efetivo - Ministério da Ciência e Tecnologia ]{Pagamento efetivo - Ministério da Ciência e Tecnologia (Fonte: Feito pelo autor)}
    \label{fig:quedaorca1}
\end{figure}
\begin{figure}[!h]
    \centering
    \includegraphics[width=0.8\textwidth]{04-figuras/compartive_last .png}
    \caption[Pagamento efetivo - Relativos aos anos anteriores ]{Pagamento efetivo - Relativos aos anos anteriores (Fonte: Feito pelo autor)}
    \label{fig:quedaorca}
\end{figure}

Assim, considerando os impactos econômicos e sociais apresentados, o presente trabalho se propõe a discutir alternativas e abordagens que amenizem impactos de orçamento disponibilizados pelo governo, para a viabilização de projetos de pesquisa que utilizam bases de dados tão grandes.

Nesse sentido, temos dois efeitos diretos:

\begin{itemize}
    \item Redução de fragilidade orçamentária; e
    \item Proposta de alternativas para análise de dados em saúde.
\end{itemize}

A redução de fragilidade orçamentária se dá pela proposta de utilização de recursos já disponíveis e subutilizados, como \emph{desktops} de bibliotecas, laboratórios etc, por meio de ambientes virtualizados, compartilhando esses recursos com a orquestração das cargas de trabalhos de análise de dados. Para viabilizar essa proposta, o trabalho compara a performance de sistemas virtuais em computadores com baixo poder computacional, propondo assim uma forma de abordar, tanto o isolamento, para processos mais sensíveis, quanto a utilização de recursos que já estão disponíveis, a fim de garantir o menor CAPEX possível para excussão do projeto de pesquisa. 

Para avaliação dessa alternativa toma-se como exemplo os bancos de “Vendas de Medicamentos Controlados e Antimicrobianos - Medicamentos Industrializados”, objeto de análise deste trabalho. Nesse banco estão disponíveis cerca de 70 GB e mais de 500 milhões de linhas de dados sobre a comercialização de medicamentos de venda controlada no país. Logo, tão importante quanto a disponibilidade pública dos dados, é fundamental encontrar estratégias técnicas e economicamente viáveis a fim de possibilitar que pesquisadores em todo o país possam contribuir com a análise e a interpretação desses dados, mesmo frente a baixa disponibilidade de recursos financeiros e de infraestrutura, como servidores de alta performance (HPC), por exemplo.

\section{Objetivos}
\label{sec:objetivos}
Diante disso, com a realização deste trabalho espera-se oferecer uma alternativa para análise de grandes volumes de dados que possua baixo custo financeiro, menor complexidade de configuração, maior efetividade (menor tempo de análise) e que não seja dependente da disponibilidade ou uso de recursos dedicados, como computadores de alta performance (HPC). A abordagem que irá se seguir é na utilização de alternativas \emph{open source} atualmente disponíveis e que sejam possíveis de serem utilizadas em computadores de baixo custo e menor poder computacional - ex.: \emph{OpenStack}, \emph{CloudStack} etc.

Deste modo, espera-se demonstrar comparativamente a implementação de uma solução para análise de dados em plataformas de orquestração de \emph{containers}, que permita recrutar computadores comuns para essa análise. E assim, espera-se, superar de maneira custo-efetiva um problema de restrição orçamentária e técnica para instituições públicas e grupos de pesquisa que realizam análises de grande volumes de dados. No caso deste trabalho, a aplicação está direcionada para a área da saúde, utilizando uma tecnologia já amplamente empregada no setor privado, o que viabiliza o suporte de estudantes e/ou profissionais das áreas de Engenharias e Computação. Espera-se, ainda, contribuir para que os dados públicos em saúde sejam analisados com  maior frequência e menor restrição, gerando indicadores melhores e atualizados para melhor tomada de decisão em saúde.

\section{Definição e abordagem}
\label{sec:abordagem}

A proposta do trabalho visa avaliar a utilização de \emph{cluster} de Kubernetes\textregistered \ como plataforma de orquestração de cargas de trabalho para processamento de dados em \emph{clusters }compostos por computadores de baixo custo ou reaproveitados.  Utilizando conceitos e ferramentas DevOps (BASS et al, 2015) intenta-se abordar o provisionamento, a integração e o \emph{deploy} da infra estrutura, valendo-se da automação em esteiras de CI (continuous integration), CD (continuous delivery). Utiliza-se também de ferramentas de IaC (Infrastructure as Code), que tornam a configuração e disponibilização desse \emph{cluster} mais ágil, auto documentada e diminui a necessidade de operação e manutenções no \emph{cluster},  outros recursos e aplicações configuradas para a solução proposta nesse trabalho.

A utilização da plataforma visa validar seu uso para orquestração de tarefas em paralelo e processamento distribuído de dados durante a análise, permitindo o uso simultâneo de diversas máquinas. Inicialmente, foram utilizados 8 computadores com capacidades de processamento semelhantes a computadores \emph{desktop} de 8-16 GB (Gigabytes) de RAM (Random Access Memory) e 4-8 vCPU (virtual Central Process Unit). Essa restrição  permitiu uma análise de viabilidade para que o processamento e análise de grandes massas de dados (maiores que 50 GB) possam ser feitas sem o uso de HPC.

Utilizando como carga de trabalho a análise de tendência de consumo de azitromicina no Brasil, entre os anos de 2014 e 2021. Pretende-se utilizar, como principal resultado, a viabilidade de utilização de computadores do tipo \emph{desktop} para orquestração e análise de grades massas de dados como dados do SUS (Sistema Único de Saúde) garantindo assim viabilidade de trabalhos e proponto a utilização dessa plataforma em computadores menos específicos, como alternativa a HPC.

O trabalho não engloba a realização de interpretação da informação gerada pelo banco, garantindo, assim, apenas o resultado correto da análise citada como carga de trabalho para comparação. Também não está sendo proposta uma metodologia de análise do banco referenciado, mas a avaliação das tecnologias empregadas para orquestração das tarefas, comparação de desempenho entre as alternativas da implementação da plataforma e sua implementação como proposta para uso mais amplo nas instituições sob restrição orçamentária, com o fim de continuar a realizar análises de dados, ainda que sem hardware adequado. 

As principais métricas visadas para avaliar a viabilidade do \emph{cluster} são tempo de execução da atividade de importação e processamento do banco, taxa de utilização de CPU e memória do \emph{cluster}, garantido que o mesmo permaneça disponivel para outras cargas de trabalho, i.e não ocupando toda a capacidade disponivel, para que ainda possam ser executadas atividades simultaneas no \emph{cluster} não relacionadas a análise. Métricas complementares, com taxa de leitura e escrita em disco e tráfego de rede são coletadas e discutidas, mas não são o principal objetivo desse trabalho, sendo que não considera-se aqui comparação com diferentes tipos de protocolos para armazenamento. Sendo que protocolos diferentes de armazenamento compartilhado ou distribuidos em redes podem afetar diretamente na taxa de utilização da rede. 


\section{Organização do trabalho}
\label{sec:organizacaoTrabalho}

Este trabalho está dividido em cinco capítulos. Na Introdução foi apresentado o o contexto geral do trabalho, tecnologias que são avaliadas e possíveis problemáticas econômicas e sociais. 

O Capítulo \ref{chap:fundamentacaoTeorica} apresenta a fundamentação teórica e a revisão da literatura, discorrendo sobre as leituras que embasaram toda a construção do projeto. Nele também são aprofundados em conceitos necessários para a discussão e condução do trabalho. 

O Capítulo \ref{chap:metodologia} apresenta a  metodologia utilizada para a construção dos componentes do projeto e a forma de avaliação de desempenho dos ambientes propostos, bem como a forma de avaliação. 

O Capitulo \ref{chap:resultados} apresenta os resultados obitidos tanto para orquestração das cargas de trabalho no ETL (extração, transformação e carga) e também os resultados obtidios apara a análise e processamento dos dados de consumo e prescrição obtidos para azitromicina, no periodo descrito anteriormente.
São discutidos os resultados obtidos para a orequestração do \emph{cluster} e na análise dos dados de prescrição da azitromicia, além de suas interpretações. 

O Capítulo \ref{chap:conclusao} apresenta as conclusões do trabalho, incluindo o que o aluno realizou ao longo da execução do trabalho, se a proposta do trabalho atingiu a espectativa como solução ao problema  proposto, além de possibilidades para trabalhos futuros. 