% -----------------------------------------------------------------------------
% Abstract
% -----------------------------------------------------------------------------

\begin{resumo}[Abstract]
    The analysis of large volumes of data sometimes requires one or more high-performance computers to make extracting information viable. In the research's area, mainly in public institutions, there is a budget constraint that, in recent years, has shown a growing decrease in available resources, making it impossible to acquire computers that allow such analyses. An alternative to the need for high-performance computers is the use of distributed \emph{computing} techniques. Thus, the orchestration of clustered resources with Kubernetes\textregistered\ was performed for data processing and analysis. In addition, the feasibility of this solution was evaluated in terms of performance, complexity and economic viability, having as a workload the analysis of azithromycin consumption data, in Brazil, between the years 2014 and 2020. Eight computers were used in the structuring the cluster, adding up to 42 CPUs and 88 GiB RAM. For ease of configuration while keeping the \emph{clusters} secure, they were all configured with SSH keys for remote access. The use of \emph{desktops} in the composition of the \emph{cluster} - which will continue to be available for use in the Department of Computer Science - made it possible to process more than 70 GB of data in a distributed manner, in 53 minutes, without any additional cost with the acquisition of equipment for analysis of data. Regarding the data on azithromycin consumption, in Brazil, an increase in prescription rates per 1,000 inhabitants was observed over the time series from 66.2 to 83.8 prescriptions, as well as in almost all federative units, with emphasis on Minas Gerais (66.2 to 149.2 prescriptions per 1,000 inhabitants), Rondônia (37.5 to 109.4 prescriptions per 1,000 inhabitants) and Roraima (37.2 to 99.8 prescriptions per 1,000 inhabitants). A similar behavior was also observed, in the comparative analysis between the years before and during the pandemic (2019 and 2020), with an increase from 59.9 to 83.8 prescriptions per thousand inhabitants, in the country. The accomplishment of this work showed that the use of Kubernetes\textregistered\ \emph{cluster} is a promising alternative for analyzing large volumes of data, especially from the point of view of using underutilized resources, with the recruitment process through dynamic configuration of configuration managers. And that, considering the data analyzed, apparently the COVID-19 pandemic - including the disclosure of azithromycin in COVID kits - may have influenced the greater increase in the consumption of this drug.


    \textbf{Keywords}: Kubernetes. Virtualization. Containers. Hypervisor Type 2. Data analysis.
\end{resumo}

% -----------------------------------------------------------------------------
% O restante da formatação deve manter-se igual ao do resumo em português, i.e, um único parágrafo.
% -----------------------------------------------------------------------------
